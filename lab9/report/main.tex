\documentclass[a4paper,14pt]{article}
\usepackage{blindtext}
\usepackage[T2A]{fontenc}
\usepackage[utf8]{inputenc}
\usepackage[english,russian]{babel}
\usepackage{listings}
\usepackage{geometry}
\usepackage{amssymb}
\usepackage{amsmath}
\usepackage[14pt]{extsizes}
\geometry{left=3cm}
\geometry{right=1.5cm}
\geometry{top=2cm}
\geometry{bottom=2cm}
\pagestyle{plain}
\usepackage{pgfplots}
\usepackage{filecontents}
\usepackage{graphicx}
\usepackage{indentfirst}
\DeclareGraphicsExtensions{.png}
\graphicspath{{images/}}
\usetikzlibrary{datavisualization}
\usetikzlibrary{datavisualization.formats.functions}
\usepackage{tabularx}
\pgfplotsset{width=7 cm}
\usepackage{xcolor}
%\renewcommand{\rmdefault}{ftm}
%\usepackage{mathptmx}
\usepackage{setspace}
%\usepackage{minted}
%\полуторный интервал
\onehalfspacing
\usepackage{colortbl}
\usepackage{tocloft}
\frenchspacing
\setcounter{page}{3}
\usepackage{multirow}
\usepackage{float}
\usepackage{multirow}

\renewcommand{\cftsecdotsep}{\cftdot}
\renewcommand{\cftsecleader}{\cftdotfill{\cftsecdotsep}}
\renewcommand{\cftsubsecleader}{\cftdotfill{\cftsecdotsep}}
\renewcommand{\cftsubsubsecleader}{\cftdotfill{\cftsecdotsep}}

%\renewcommand\cftchapdotsep{\cftdot}
%\renewcommand\cftsecdotsep{\cftdot}
%\renewcommand{\cftchapleader}{\cftdotfill{\cftchapdotsep}}

% Для измененных титулов глав:
% % подключаем нужные пакеты
%\definecolor{gray75}{gray}{0.75} % определяем цвет
%\newcommand{\hsp}{\hspace{20pt}} % длина линии в 20pt
% titleformat определяет стиль
%\titleformat{\chapter}[hang]{\Huge\bfseries}{\thechapter\hsp\textcolor{black}{|}\hsp}{0pt}{\Huge\bfseries}
%\usepackage{titlesec, blindtext, color}
%\titleformat{\chapter}[hang]{\Huge\bfseries}{\thechapter\hsp\textcolor{black}{|}\hsp}{0pt}{\Huge\bfseries}

% Для листинга кода:
\lstset{ %
extendedchars=\true,
inputencoding=utf8,
morekeywords={include, printf},
texcl=\true,
breaklines=\true,
escapeend=\end{russian},
escapechar=\%,
keepspaces=\true,
language=c,                 % выбор языка для подсветки
basicstyle=\small\sffamily, % размер и начертание шрифта для подсветки кода
numbers=left,               % где поставить нумерацию строк (слева\справа)
numberstyle=\tiny,           % размер шрифта для номеров строк
stepnumber=1,                   % размер шага между двумя номерами строк
numbersep=5pt,                % как далеко отстоят номера строк от подсвечиваемого кода
showspaces=\true,            % показывать или нет пробелы специальными отступами
showstringspaces=\true,      % показывать или нет пробелы в строках
showtabs=false,             % показывать или нет табуляцию в строках
frame=single,              % рисовать рамку вокруг кода
tabsize=4,                 % размер табуляции по умолчанию равен 2 пробелам
captionpos=t,              % позиция заголовка вверху [t] или внизу [b]
breaklines=true,           % автоматически переносить строки (да\нет)
breakatwhitespace=false, % переносить строки только если есть пробел
escapeinside={\//*}{*)}   % если нужно добавить комментарии в коде
}



\begin{document}

\begin{titlepage}

    \begin{table}
        \centering
        \footnotesize
        \begin{tabular}{cc}
            \multirow{8}{*}{\includegraphics[scale=0.35]{bmstu.jpg}}
             &                                                                           \\
             &                                                                           \\
             & \textbf{Министерство науки и высшего образования Российской Федерации}    \\
             & \textbf{Федеральное государственное бюджетное образовательное учреждение} \\
             & \textbf{высшего образования}                                              \\
             & \textbf{<<Московский государственный технический}                         \\
             & \textbf{университет имени Н.Э. Баумана>>}                                 \\
             & \textbf{(МГТУ им. Н.Э. Баумана)}                                          \\
        \end{tabular}
    \end{table}

    \vspace{-2.5cm}

    \begin{flushleft}
        \rule[-1cm]{\textwidth}{3pt}
        \rule{\textwidth}{1pt}
    \end{flushleft}

    \begin{flushleft}
        ФАКУЛЬТЕТ Информатика и системы управления
    \end{flushleft}
    КАФЕДРА Программное обеспечение ЭВМ и информационные технологии

    \vspace{3cm}

    \begin{center}
        \textbf{Лабораторная работа № 9} \\
        \textbf{Дисциплина: <<Компьютерные сети>>}
        \vspace{0.5cm}
    \end{center}


    \vspace{3cm}

    \begin{flushleft}
        \begin{tabular}{ll}
            \textbf{Студент}       & Овчинникова А. П. \\
            \textbf{Группа}        & ИУ7-75Б           \\
            \textbf{Преподаватель} & Рогозин Н. О.     \\
        \end{tabular}
    \end{flushleft}

    \vspace{3cm}

    \begin{center}
        Москва, 2020 г.
    \end{center}

\end{titlepage}

\setcounter{page}{2}

Пример назначения адреса Server 0 из подсети 1 представлен на рисунке \ref{fig:server0}.

\begin{figure}[!h]
    \center{\includegraphics[width=14cm]{server0}}
    \caption{Настройка Server 0.}
    \label{fig:server0}
\end{figure}

\textbf{Настройка VLan на коммутаторе 2 уровня}

Последовательность команд для настройки коммутатора:

\begin{lstlisting}
en
conf t
interface vlan 30
interface range FastEthernet0/5-FastEthernet0/7
switchport mode access
switchport access vlan 30

interface vlan 20
interface range FastEthernet0/3-FastEthernet0/4
switchport mode access
switchport access vlan 20

interface vlan 10
interface range FastEthernet0/1-FastEthernet0/2
switchport mode access
switchport access vlan 10
\end{lstlisting}

С помощью команды \textit{show vlan} можно проверить результат (\ref{fig:sw_vlan}).

\begin{figure}[!h]
    \center{\includegraphics[width=14cm]{sw_vlan}}
    \caption{Настройка Switch 0.}
    \label{fig:sw_vlan}
\end{figure}

\textbf{Настройка VLan на коммутаторе 3 уровня/маршрутизаторе}

Последовательность команд для найстройки VLan на маршрутизаторе:

\begin{lstlisting}
en
conf t

int g0/0/0.1
encapsulation dot1q 10
exit

int g0/0/0.2
encapsulation dot1q 20
exit

int g0/0/0.3
encapsulation dot1q 30
exit
\end{lstlisting}

На рисунке \ref{fig:sch} выделена и озаглавлена каждая виртуальная локальная сеть.

\begin{figure}[!h]
    \center{\includegraphics[width=14cm]{sch}}
    \caption{Виртуальные локальные сети.}
    \label{fig:sch}
\end{figure}


\end{document}
